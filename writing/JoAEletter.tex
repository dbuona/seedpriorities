
\documentclass[11.5 pt]{article}
\usepackage[margin=1.2in]{geometry}
\usepackage{graphicx}
\usepackage{natbib}
\usepackage{gensymb}
%\begin{footnotesize}
%\address{1300 Centre Street \\ Boston, MA, 20131}
%\end{footnotesize}

\usepackage{Sweave}
\begin{document}
\input{JoAEletter-concordance}
\bibliographystyle{..//refs/styles/besjournals.bst}
\def\labelitemi{--}
\parindent=24pt
\noindent\includegraphics[width=0.2\textwidth]{/Users/danielbuonaiuto/Desktop/AA_logo.jpg}
\pagenumbering{gobble}
\\\\
\noindent{Dear Drs. Barlow, Pettorelli, Stephens,  Nuñez, Rader and Siqueira,}\\
\vspace{1.5ex}

\noindent Please consider this manuscript, ``Seedling competition with an invasive woodland herb is mediated by cold stratification" as a research article in \textit{Journal of Applied Ecology}.\\

\noindent As humans move organisms around the globe at unprecedented rates, the question:  \textit{``Which traits facilitate the success of introduced species?"} is critical to the detection and management of high impact invaders \citep{Fournier2019}. 
Rapid germination and precocious phenology are traits common to many invasive plant species. These traits have received substantial attention because the ``head start" provided by early germination may function as a seasonal priority effect, giving invaders a competitive advantage over slower-germinating natives \citep{Wainwright_2011}. Further, because germination phenology is closely linked to environmental cues, the impact of seasonal priority effects may be exacerbated by climate change \citep{Rudolf:2019aa}. Despite the growing interest in seasonal priority effects, it has been difficult to quantify their overall contribution to the competitive success of invaders, and few studies to date has mechanistically linked climate to priority effects. Instead most studies use artificially staggered planting or sowing to create priority effects \citep{Young:2017aa}.\\

\noindent We address these questions about the importance of seasonal priority effects in a changing climate using a suite of controlled environment experiments to vary seasonal priority effects via environmental variation. We first evaluated how climate variation impacted the germination behavior of two herbaceous woodland species---one invasive to North America and one native. We then leveraged these differences to manipulate the relative germination phenology of these taxa to quantify the effects of phenological differences on competitive outcomes. Our results show that precocious germination doubled the competitive impact of the invader relative to its other intrinsic competitive traits, and that phenological advances of just two to three days, relative to the native competitor, were enough to secure competitive dominance at the seedling stage.\\

\noindent Our approach allowed us to mechanistically link climate variation, phenology and competitive dynamics. This is an important advance over previous approaches because it can be directly translated into forecasting patterns of phenological assembly with climate change. These results and conceptual advances have the potential to impact a number of applications from the management and restoration of ecological communities to conservation planning.\\

\noindent The main text of this manuscript is 5977 words in length and it contains three figures. It is co-authored by E.M. Wolkovich, and is not under consideration elsewhere. We hope that you will find it suitable for publication in \textit{Journal of Applied Ecology}, and look forward to hearing from you.\\\\ 
\\Sincerely,\\\\\\\\\\

\noindent Daniel Buonaiuto\\

\bibliography{..///refs/germination.bib}

\end{document}
