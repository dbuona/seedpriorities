\documentclass[11pt]{article}
%Required: You must have these
\usepackage[margin=.85in]{geometry}
\usepackage{graphicx}
\usepackage{tabularx}
\usepackage{natbib}
\usepackage{pdflscape}
\usepackage{array}
\usepackage{authblk}
\usepackage{gensymb}
\usepackage{amsmath}
%\usepackage[backend=bibtex]{biblatex}
\usepackage[small]{caption}

\setkeys{Gin}{width=0.8\textwidth}
\setlength{\captionmargin}{30pt}
\setlength{\abovecaptionskip}{10pt}
\setlength{\belowcaptionskip}{10pt}

\topmargin -1.5cm 
\oddsidemargin -0.04cm 
\evensidemargin -0.04cm 
\textwidth 16.59cm
\textheight 21.94cm 
\parskip 7.2pt 
\renewcommand{\baselinestretch}{1} 	
\parindent 0pt
\usepackage{setspace}
\usepackage{lineno}
\bibliographystyle{..//..//sub_projs/refs/styles/besjournals.bst}
\usepackage{xr-hyper}
%\usepackage{hyperref}
\externaldocument{
\externaldocument{p}

\begin{document}
\emph{Reviewer comments are in italics.} Author responses are in plain text.\\

\textbf{Handling Editor Comments for Authors:}

\emph{This paper has received two consistent external reviews. Both reviewers applaud the question and topic of the paper, which is centered on how priority effects might contribute to the success of invasive plants in terms of germination phenology. This is a relevant question in the context of climate change and modern coexistence theory. The main result is that the native species is a better competitor in the absence of any phenological advantage; however, with only one day of germination advantage, the invasive species becomes superior. The main concern with the paper is that it is vague about whether the two study species ever actually co-occur in nature, so it is unclear how much relevance this result has for natural systems (I also shared this concern with reviewers). Without more information on co-occurrence of the study species, it is not clear whether the paper has a broad enough reach for Journal of Ecology. Another concern that brings into question the relevance of the result for natural systems is the fact that the two study species have different types of dormancy. Please thoroughly address these concerns and others raised by the reviewers. }

We are grateful to the Handling Editor and two Reviewers for their comments and feedback on our initial submission. With your input, we feel that we have much improved this manuscript. In particular, have added considerable text about our process of species selection and new text and analyses sharpening our presentation of what is known about the interactions between these species in the natural systems. We have also clarified how studying species with contrasting dormancy types in the context of our study was an intentional decision, and increases the relevance of this study to applications in real communities.  The changes are detailed below.

\emph{Line 100: Please elaborate more about this. Have you ever observed them co-occurring together, or can you find some GIS layers, iNaturalist occurrence data, or something similar to further suggest that they co-occur?} 

We thank the editor for raising this point. One of reasons that went into choosing these species for our study was our personal observations of them growing together, and we now can see that we did not clearly present evidence of their interactions in natural environments in our inital version of thios manuscript. We also thank the editor for their suggestions for potential data sources to verify our assertion these species co-occur. In this version of the manuscript we have added two new data presentations to demonstrate that our focal species frquently co-occur in natural system.

First, we extracted all geo-located North America observations of the two species from GBIF (X observation total). We used these data to generate a map, which we now include as a new figure in the revised version of the manuscripts, that highlight that species frequently occur in proximity to each other.

Second, we queired a recently published database that compiles over 80,000 observational plots of plant communitites from federal and state agencies across the United States for plots in which these species co-occur. Of the 160 plots in the temperate forest region in which \emph{H. matronalis} was detected, \emph{Cryptotaenia canadensis} was present at 24\% of them. 

We now present these two examples in a broader ``Species selection" sections of our Methods (lines \lineref{occ1}-\lineref{occ2} ).


\emph{On a related note, why were the seeds collected from different habitats, and what implications does this have for interpretation of the results?}

We appreciate this concern. Restrictions to research activity during the covid-19 pandemic neccisated that we purchase our seeds rather than collect them ourselves. This meant that matching our source environments was restricted by commercial availability of plant materials. 

As the Editor points out, it is clear from the literature that site specific factors (e.g.  transgenerational plasticity, local adaptation, etc) could influence the patterns we observed in our study, and we could have explained this better in our original submission. In the updated version of the manuscript, we address  this. TO DO

\emph{Line 153: change ‘and’ to ‘an’}

Thanks. We have made this change.

\emph{Line 215: I would save this interpretation for the Discussion section.}

We have removed our interpretion of this result form this section. This statement is now paraphased in the Discussion at line \lineref{mov1}

\emph{Line 233:  I’m not convinced you can say this unless you know that these species actually co-occur. Otherwise, it remains unknown how important this actually is in nature. Same for lines 237–239. Additionally, another caveat is that we don’t know about all of the other species interactions that would occur in a natural setting, which could modify the pairwise effects of this study.}

We hope that the additional analyses we include in our revised submission provide stronger support for these assertions. The editor's point about how presence additional species could modify the interactions between our focal species is well taken. We made serveral changes to our text to better articulate what the limitations of our approach and research directions needed to robustly apply our results to natural systems.

We have added the caveat.


One potentially relevant paper that was not cited is Kimball et al. (2010). Contemporary climate change in the Sonoran Desert favors cold-adapted species. Global Change Biology 16: 1555-1565

----------------------

Reviewers’ comments:

Reviewer: 1

COMMENTS FOR THE AUTHOR
\emph{The manuscript contains good information. But the problem statement is not done well. Why should this study be done? Is this invasive species a problem for the native species? Is there a report on this? In the materials and methods section, I saw that you wrote that the seeds were collected from two different habitats. If they have different habitats, why should the competition between them be compared?! Another question I have is that these two species differ in their type of dormancy. One has physiological dormancy and the other morphological/morphophysiological dormancy. How would you expect these two species not to differ in terms of germination, especially germination speed and response to stratification. You can see some of my other comments in the attached file. With these problems I have to reject the manuscript.}

We thank the Reviwer for the comments are are pleased they found the information in this study to be useful. The Reviewer provided us with serveral imporant questions that we have addressed in our new version of the manuscrpt that we feel has resulted in a much improved submission. We've seperated out each questiona and resposes below for clarity.

\emph{The manuscript contains good information. But the problem statement is not done well. Why should this study be done?} 

Based on their feedback and guidance in the attached file, we have re-written the concluding paragraph of our Introduction to more clearly articulate the questions this study adresses as well the scope of inference, assumptions amnd limitations of our design and inference in lines \lineref{prob1}-\lineref{prob2}. Our problem statement now reads:
\begin{quote}
In this study, we\linelabel{prob1} perform a sequences of experiments in controlled environments to link climate variation, phenological advantage, and seasonal priority effects to the competitive interactions of two woodland herbaceous species (the invasive \textit{Hesperis matronalis} and native \textit{Cryptotaenia canadensis}) that frequently co-occur in the understory of temperate forest regions of North America. First, we performed a series of germination assays under varying temperature regimes to address the question: 
\begin{quote}How does phenological advantage between two species with contrasting seed dormancy types shift in response to variable climate conditions?\end{quote}
We then used competition trials under contrasting environmental conditions to indirectly manipulate the phenological advantage between these two taxa to address a second question: \begin{quote}To what extent do seasonal priority effects generated by varying patterns of phenological advantage influence the competitive dynamics of seedlings?\end{quote}
\end{quote}

We also include information about the assumptions of our approach and the potential scope of inference from our finding in lines \lineref{cav1}-\lineref{prob2}. 

\emph{Is this invasive species a problem for the native species? Is there a report on this?}

This question was shared by both reviewers and the editor. We have addressed this question by providing a new figure demonstrating high degrees of overlap in patterns of occurence amoung these species, and analyses of plant community data indicating the frquently interaction (lines \lineref{occ1}-\lineref{occ2}. These changes have been detailed more fully above in the responses to the editor's comments.

\emph{In the materials and methods section, I saw that you wrote that the seeds were collected from two different habitats. If they have different habitats, why should the competition between them be compared?!}

We thank the reviewer for highlighting this point, which was a concern also raised by the editor. The covid-19 pandemic necessitated that we purchase our seeds, and the seeds of these species were not available from the same providers. We have elaborated on how using different seed sources may impact the inference from our study in lines.  With that said, broad patterns patterns germination and phenological responses to the environment to be conserved at the species level. TO DO


\emph{Another question I have is that these two species differ in their type of dormancy. One has physiological dormancy and the other morphological/morphophysiological dormancy. How would you expect these two species not to differ in terms of germination, especially germination speed and response to stratification.}

We agree with the reviewer here--- we specifically choose these species because we expected them to have different responses to stratification. 
Our study was designed to specifically leverage these differences to generate the sufficent amount of variation in phenological advatange, which was one of the main cruxes of our experiment. In our re-submission we have added new text in serveral places to emphasize that this was an intentional choice that enhances the relevance of the study rather than a limitation.

First, we include new citations that highlight that species with different dormancy types often co-occur in nature (line).

Second we explain in more detail in our Methods section about why this choice is, infact, critical to our experimental design (lines \lieref{crit3}-\lineref{crit4}).

Third, we highlight how the fact that our study directly shows that there are a range of environmental conditions under which these two species with different dormancy type display nearly identical germination behavior demonstrates that the paradigm that dormancy types serve to maintain differences in expressed germination patterns amoung species is an over-simplification of how species experience environmental cues. We have address this in text with the follwing addition to our Discussion (line):


\textbf{Reviewer: 2}

\emph{The present article studies de effects of temperature variation, due to climate change, which may affect germination phenology and competitive interactions between native and introduced plant species under controlled conditions. This temperature variation may favor the germination and establishment success of introduced species, which may affect the composition of native plant communities.

I really appreciate the focus on priority effects, climate change, and introduced vs. native plant species. I admit that when first reading the title, I did not think about all this topics together. However, they are well presented in the article.}



\emph{This article has the potential to significantly contribute to the scientific literature. However, it will first require some changes to clarify:}


\emph{1) It is not clear how the authors selected the focal species. The authors need to justify their decision.}

\emph{2) The title does not match the content}

We have changed the title.


\emph{Other comments:}

\emph{Title: I suggest you to improve it, because when reading it first I did not get the idea that it will talk about warming climate, priority effects, and native vs. invasive species.}


\emph{ Abstract:} 

\emph{L22: “germination phenology for two North American herbaceous species” makes me think that you are going to study 2 native species. Afterwards, it says the invasive and the native, then it makes me guess if the invasive is introduced/exotic or not. Please make it clear.}


\emph{L16: what are other intrinsic competition traits can you mention?}


Body:
\emph{L55-57: I suggest characterized instead of dispersed.}

\emph{L82: “the North American invasive” for me this sounds like Hesperis matronalis is native to North America and that does not belong to the habitat of Cryptotaenia canadensis, due to that it is invasive.}


\emph{L100-103: You mention that it is known that C. canadensis need a substantial period of cold moist stratification to release dormancy and germinate. Also you said that they have different germination niche. Then, if all this is known why do we need to run an experiment to test it? Please justify better the slection of these focal species.}

We thank the reviewer for highlighting this point. Our study intentionally leveraged species with different dormancy types and responses to the environment, to generate the sufficent amount of variation in phenological advatange to run a meaningful experiment. We can see that we did not make this clear in our initial submisison, and have made several changes to our method second to highlight this.

First, we have changed the "Focal species" sub-section of out methods to be "Species selection" and more comprehensively laid out our critiera for choosing species in lines \lineref{crit1}-\lineref{crit2}.

Second, we have added a paragraph to explain both why including species with a different dormancy types is was critical instrumental choice for our study, and its relevance to natural systems (lines \lineref{crit3}-\lineref{crit4}).

We feel these changes, combined with the inclusion of the new analyses demonstrating our focal species often interact in nature that we detailed above, substantially improved our presentation of context of this study, and we are grateful to the reviewer for pushing us to do this.

\emph{L104: “suggest” instead of suggestS}

Thank you. We've made this change.

\emph{L112: remove one OF}

Thank you for catching this typo. It is corrected in the updated submission

L113-114: The authors use low or cool incubation temperature and high or warm incubation temperature. I would suggest deciding for one and using it consistently all over the manuscript.

L128: Please mention what you have used as substrate in your pots.

L133: I have seen that seed sizes of both species are really different, and also I have seen that they have different sizes when adults. Please justify why you are using those seed densities, in terms of natural history of your focal species, in your competition trial.

L191-192: C. Canadensis in italics.

L249: “lower stratification treatments” I would suggest you to specify, something like “less time/days/weeks of stratification”

\end{document}