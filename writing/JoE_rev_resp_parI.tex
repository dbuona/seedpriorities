\documentclass[11 pt]{article}
\usepackage[margin=.85in]{geometry}
\usepackage{graphicx}
\usepackage{natbib}
\usepackage{gensymb}
%\begin{footnotesize}
%\address{1300 Centre Street \\ Boston, MA, 20131}
%\end{footnotesize}
\begin{document}
\bibliographystyle{..//..//sub_projs/refs/styles/besjournals.bst}
\def\labelitemi{--}
\parindent=24pt
\noindent\includegraphics[width=0.2\textwidth]{/Users/danielbuonaiuto/Desktop/AA_logo.jpg}
\pagenumbering{gobble}

\noindent{Dear Dr. Hector,}\\
\vspace{1.5ex}

\noindent Please consider our revised manuscript, now titled ``Contrasting responses to climate variability generate seasonal priority effects between native and invasive forest herbs" as a ``research article" in \textit{Journal of Ecology}.\\

%emwmar31 -- check edits (including 'them' I did not fully understand -- ??)
\noindent Rapid germination is  a trait common to many invasive plant species, which may give invaders a competitive advantage over slower-germinating natives. Because the germination timing of many forest plant species is closely linked to environmental cues, the impact of these germination differences may be exacerbated by climate change. It has been difficult, however, to quantify the overall contribution of rapid germination to the competitive success of invaders, and few studies to date have mechanistically linked these kinds of competitive outcomes to climate. We addressed these questions with a series of controlled-environment experiments in which we indirectly varied the germination timing of native and invasive herbs by manipulating climate variables. We found that precocious germination doubled the competitive impact of the invader relative to its other intrinsic competitive traits; a germination advantage of just two to three days, was enough to secure competitive dominance.\\

%emwmar31 --check editors name
\noindent Comments from the Handling Editor, Dr. Amy Iler, and two reviewers suggested that the topic of our manuscript was timely and important but pointed out a need to better present our rationale for selecting our focal species and to qualify how our laboratory results translate into complex forest ecosystems.\\

\noindent Based on their comments we have made important changes to the manuscript. These changes include the addition of a new figure in the main text and an additional data table in the Supporting Information, which show that our focal species co-occur frequently in eastern forests, as well as a new sub-section in our Methods laying out the specific selection criteria we considered. We have also changed our Introduction to more clearly state the specific goals of this study, and added a new subsection to our Discussion ``A research agenda for assessing the role of seasonal priority effects in plant communities of temperate forests" to place our finding in a broader context of temperate forest community ecology. We have also changed our title as requested. \\ 

\noindent We feel that the editor's and reviewers' input has yielded a new submission that is much improved, and we detail our specific changes in the following pages with reviewer comments in \emph{italics} and our responses in regular text.\\


\noindent The main text of this manuscript is 6,147 words in length and now contains four figures. It is co-authored by E.M. Wolkovich, and is not under consideration elsewhere. We hope that you will find it suitable for publication in \textit{Journal of Ecology}, and look forward to hearing from you.\\


\noindent Best,\\\\




\noindent Daniel Buonaiuto

\end{document}

