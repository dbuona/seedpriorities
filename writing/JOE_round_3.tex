
\documentclass[11pt]{article}
%Required: You must have these
\usepackage[margin=.85in]{geometry}
\usepackage{graphicx}
\usepackage{tabularx}
\usepackage{natbib}
\usepackage{pdflscape}
\usepackage{array}
\usepackage{authblk}
\usepackage{gensymb}
\usepackage{amsmath}
%\usepackage[backend=bibtex]{biblatex}
\usepackage[small]{caption}

\setkeys{Gin}{width=0.8\textwidth}
\setlength{\captionmargin}{30pt}
\setlength{\abovecaptionskip}{10pt}
\setlength{\belowcaptionskip}{10pt}

\topmargin -1.5cm 
\oddsidemargin -0.04cm 
\evensidemargin -0.04cm 
\textwidth 16.59cm
\textheight 21.94cm 
\parskip 7.2pt 
\renewcommand{\baselinestretch}{1} 	
\parindent 0pt
\usepackage{setspace}
\usepackage{lineno}
\bibliographystyle{..//..//sub_projs/refs/styles/besjournals.bst}
\usepackage{xr-hyper}
%\usepackage{hyperref}

\begin{document}
\emph{Reviewer comments are in italics.} Author responses are in plain text. 

\emph{Handling Editor Comments for Authors:}

\emph{The authors have addressed all of the previous concerns of the reviewers and myself. I appreciate the reframing and additional content added to the manuscript and also find it much improved. This paper will be an important contribution to the literature on priority effects, especially in the context of plant invasion and climate change. I have a few, incredibly minor comments that should be addressed before the paper is accepted.} 

We are very pleased that the Handling Editor sees the improvements to our manuscript, and considers it a important contribution to the literature. We are grateful for their feedback and help shaping this manuscript. We have clarified and fixed all the points listed below. Again, we are grateful for the editor's attention and guidance on this paper.

\emph{One thing I was a little confused about in the authors’ response was that the pandemic affected the sourcing of the seeds, but on line 147 the authors state that the initial germination experiments were conducted in 2018.}

We apologize for the confusion from our previous response. It was our competition trials that were affected by the pandemic. We had intended to obtain seeds and execute our competition experiment in the summer-fall of 2020, but were only allowed back into the lab in a very limited capacity beginning in fall 2020, and traveling widely to collect seeds was not an option. Instead, we opted to use the same seed sources as we did in the 2018 germination assays, which we had originally sourced from several different nurseries in order to test many candidate species to meet the criteria we describe in the manuscript.

\emph{Line 28: Please spell out the genus name when starting a sentence with the scientific name (here and line 119).}

We have made these corrections.

\emph{Line 113: Change is to in}

We have made this change.

\emph {Line 273: Please add a citation here.}

We have added two additional citations, listed at the end of this letter.

\emph{Line 317: Change to `effects in temperate'}

Thanks for catching this. We have corrected it now.

\emph{Line 320: Replace as with at}

We have made this change.

\emph{Line 355: Here the full genus name does not need to be spelled out.}

We have corrected this as well.


\begin{thebibliography}{61}
\providecommand{\natexlab}[1]{#1}

\bibitem[{Piao \emph{et~al.}(2019)Piao, Liu, Chen, Janssens, Fu, Dai, Liu,
  Lian, Shen \& Zhu}]{Piao:2019wd}
Piao, S., Liu, Q., Chen, A., Janssens, I.A., Fu, Y., Dai, J., Liu, L., Lian,
  X., Shen, M. \& Zhu, X. (2019) Plant phenology and global climate change:
  Current progresses and challenges. \emph{Global Change Biology} \textbf{25},
  1922--1940.

\bibitem[{Yang \& Rudolf(2010)}]{Yang:2010ta}
Yang, L.H. \& Rudolf, V.H.W. (2010) Phenology, ontogeny and the effects of
  climate change on the timing of species interactions. \emph{Ecology Letters}
  \textbf{13}, 1--10.


\end{thebibliography}


\end{document}