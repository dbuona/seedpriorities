\documentclass{article}\usepackage[]{graphicx}\usepackage[]{color}
% maxwidth is the original width if it is less than linewidth
% otherwise use linewidth (to make sure the graphics do not exceed the margin)
\makeatletter
\def\maxwidth{ %
  \ifdim\Gin@nat@width>\linewidth
    \linewidth
  \else
    \Gin@nat@width
  \fi
}
\makeatother

\definecolor{fgcolor}{rgb}{0.345, 0.345, 0.345}
\newcommand{\hlnum}[1]{\textcolor[rgb]{0.686,0.059,0.569}{#1}}%
\newcommand{\hlstr}[1]{\textcolor[rgb]{0.192,0.494,0.8}{#1}}%
\newcommand{\hlcom}[1]{\textcolor[rgb]{0.678,0.584,0.686}{\textit{#1}}}%
\newcommand{\hlopt}[1]{\textcolor[rgb]{0,0,0}{#1}}%
\newcommand{\hlstd}[1]{\textcolor[rgb]{0.345,0.345,0.345}{#1}}%
\newcommand{\hlkwa}[1]{\textcolor[rgb]{0.161,0.373,0.58}{\textbf{#1}}}%
\newcommand{\hlkwb}[1]{\textcolor[rgb]{0.69,0.353,0.396}{#1}}%
\newcommand{\hlkwc}[1]{\textcolor[rgb]{0.333,0.667,0.333}{#1}}%
\newcommand{\hlkwd}[1]{\textcolor[rgb]{0.737,0.353,0.396}{\textbf{#1}}}%
\let\hlipl\hlkwb

\usepackage{framed}
\makeatletter
\newenvironment{kframe}{%
 \def\at@end@of@kframe{}%
 \ifinner\ifhmode%
  \def\at@end@of@kframe{\end{minipage}}%
  \begin{minipage}{\columnwidth}%
 \fi\fi%
 \def\FrameCommand##1{\hskip\@totalleftmargin \hskip-\fboxsep
 \colorbox{shadecolor}{##1}\hskip-\fboxsep
     % There is no \\@totalrightmargin, so:
     \hskip-\linewidth \hskip-\@totalleftmargin \hskip\columnwidth}%
 \MakeFramed {\advance\hsize-\width
   \@totalleftmargin\z@ \linewidth\hsize
   \@setminipage}}%
 {\par\unskip\endMakeFramed%
 \at@end@of@kframe}
\makeatother

\definecolor{shadecolor}{rgb}{.97, .97, .97}
\definecolor{messagecolor}{rgb}{0, 0, 0}
\definecolor{warningcolor}{rgb}{1, 0, 1}
\definecolor{errorcolor}{rgb}{1, 0, 0}
\newenvironment{knitrout}{}{} % an empty environment to be redefined in TeX

\usepackage{alltt}
\usepackage{graphicx}
\usepackage{tabularx}
\usepackage{natbib}

\usepackage{array}
\usepackage{amsmath}
%\usepackage[backend=bibtex]{biblatex}
\bibliographystyle{..//refs/styles/besjournals.bst}
\setkeys{Gin}{width=0.8\textwidth}
%\setlength{\captionmargin}{30pt}
\setlength{\abovecaptionskip}{10pt}
\setlength{\belowcaptionskip}{10pt}
 \topmargin -1.5cm 
 \oddsidemargin -0.04cm 
 \evensidemargin -0.04cm 
 \textwidth 16.59cm
 \textheight 21.94cm 
 \parskip 7.2pt 
\renewcommand{\baselinestretch}{1.2} 	
\parindent 0pt
\renewcommand{\thetable}{S\arabic{table}}
\renewcommand{\thefigure}{S\arabic{figure}}
\usepackage{xr}
\usepackage{xr-hyper}
%\usepackage{hyperref}
\externaldocument{invasive}

\title{Supporting Information: Competition between native Honewort (\textit{Cryptotaenia canadensis}) and invasive Dame's Rocket (\textit{Hesperis matronalis}) seedlings is mediated by relative germination timing}
\IfFileExists{upquote.sty}{\usepackage{upquote}}{}
\begin{document}
\maketitle
\section{Figures}
\begin{figure}[h!]
    \centering
\includegraphics[width=\textwidth]{..//figure/AFTall.jpeg}
   \caption{The effects of weeks of cold stratification at 4\degree C on the time to 50\% germination of 11 herbacious perennials under a) cool and b) warm (20/10\degree C vs. 25/15\degree C day/night) incubation conditions, estimated with accelerated failure time model. The solid lines indicated indicated the mean estimate, while lighter line depict uncertainly with 100 random draws from the posterior distribution.} 
   \label{fig:AFTall}
\end{figure}


\begin{figure}[h!]
    \centering
\includegraphics[width=\textwidth]{..//figure/priority_treat.png}
   \caption{Differences in mean germination time between \textit{Hesperis matronalis} and \textit{C. canadensis} under 6 and 12 weeks of cold stratification.} 
   \label{fig:MGTsup}
\end{figure}

\section{Tables}
\begin{table}[ht]
\centering
\begin{tabular}{|l|l|l|l|}
\hline
species & seed source & status & seed dormancy \\
\hline
\textit{Anemone virginiana} & Prairie Moon & native & \\
\textit{Asclepias syriaca} & Toadshade & native & \\
\textit{Carex grayi} & Prairie Moon & native & \\
\textit{Cryptotaenia canadensis} & Prairie Moon & native & \\
\textit{Eurybia divaricata} & Toadshade & native & \\
\textit{Hesperis matronalis} & American Meadows & invasive & \\
\textit{Oenethera biennis} & Toadshade & native & \\
\textit{Persicaria virginiana} & Prairie Moon & native & \\
\textit{Silene stellata} & Prairie Moon & native & \\
\textit{Silene vulgaris} & wild collected & invasive & \\
\textit{Thalictrum diocicum} & Prairie Moon & native & \\
\hline

\end{tabular}
\caption{Species information for germination assays. Seed were source from a) Prairie Moon Nursery, Winona, MN b) Toadshade Wildflower Farm, Frenchtown, NJ, c) American Meadows, Shelburne VT, or d) wild collected in unmanaged section of the Arnold Arboretum, Boston MA. Dormancy catagorizations are from \citet{Baskin2015} }
\label{tab:specs}
\end{table}

\begin{table}[ht]
\centering
\begin{tabular}{|rr|ll|ll|}
   \hline
     & & Max germination (\%) & &
   Mean germination time (days) & \\ 
  \hline
  Stratification & Incubation  & C. canadensis & H. matronalis & C. canadensis & H. matronalis \\ 
  \hline
0.00 & H & 0.07 (0.1) & 0.78 (0 & 15.25 (0 & 3.11 (0.6 \\ 
 0.00 & L & 0 (0) & 0.75 (0.1 & --- & 4.59 (0.7 \\ 
   \hline
 2.00 & H & 0.03 (0) & 1 (0 & 9 (1 & 2.3 (0.1 \\ 
 2.00 & L & 0.2 (0.2) & 0.82 (0.1 & 10.25 (0.3 & 2.57 (0.5 \\ 
   \hline
 4.00 & H & 0.18 (0.1) & 0.97 (0 & 9.83 (3.6 & 2.49 (0.3 \\ 
 4.00 & L & 0.58 (0.3) & 0.82 (0.1 & 11.06 (1.1 & 3.5 (0.6 \\ 
    \hline
    5.00 & H & 0.08 (0.1) & 1 (0 & 8.44 (4.7 & 2.33 (0.4 \\ 
 5.00 & L & 0.85 (0.1) & 0.9 (0.1 & 7.67 (0.5 & 2.62 (0.6 \\ 
   \hline
   6.00 & H & 0.25 (0.2) & 0.98 (0 & 13.5 (6.9 & 1.91 (0.2 \\ 
 6.00 & L & 0.77 (0.1) & 0.97 (0.1 & 8.11 (0.4 & 2.14 (0.2 \\ 
    \hline
    7.00 & H & 0.6 (0) & 0.87 (0 & 5.81 (0.2 & 2 (0 \\ 
   7.00 & L & 0.97 (0.1) & 1 (0 & 6.29 (0.2 & 2.15 (0.2 \\ 
    \hline
    8.00 & H & 0.5 (0.1) & 1 (0 & 7.4 (0.3 & 2.06 (0.2 \\ 
   8.00 & L & 0.98 (0) & 0.95 (0 & 6.09 (0.4 & 1.94 (0.1 \\ 
      \hline
   9.00 & H & 0.6 (0.1) & 0.98 (0 & 5.22 (0.7 & 1.74 (0.1 \\ 
   9.00 & L & 1 (0) & 0.93 (0.1 & 6.04 (0.5 & 1.78 (0 \\ 
      \hline
   11.00 & H & 0.73 (0.2) & 0.98 (0 & 4.61 (0.2 & 1.86 (0.1 \\ 
   11.00 & L & 0.93 (0.1) & 0.93 (0.1 & 5.04 (0.3 & 2.11 (0.5 \\ 
      \hline
   13.00 & H & 0.77 (0.2) & 0.88 (0 & 4.14 (0.3 & 1.89 (0.9 \\ 
   13.00 & L & 1 (0) & 0.98 (0 & 4.16 (0.2 & 1.42 (0.3 \\ 
   \hline
\end{tabular}
\caption{Max germination percentages and mean germnation time for our species under all experimental treatment combination. H/L incubation (25 or 20\degree C) and weeks of chilling}
\label{tab:germcomps}
\end{table}




\begin{table}[ht]
\centering
\begin{tabular}{rrrrrrr}
  \hline
 & Estimate & Est.Error & Q2.5 & Q25 & Q75 & Q97.5 \\ 
  \hline
Intercept & 2.59 & 0.25 & 2.10 & 2.41 & 2.76 & 3.09 \\ 
  n\_Cc & -0.41 & 0.03 & -0.47 & -0.43 & -0.38 & -0.34 \\ 
  n\_Hm & 0.12 & 0.03 & 0.07 & 0.11 & 0.14 & 0.17 \\ 
  priority & 0.15 & 0.03 & 0.08 & 0.13 & 0.17 & 0.21 \\ 
   \hline
\end{tabular}
\caption{Estimates from the RGRD models}
\label{tab:RGRD}
\end{table}

\end{document}
